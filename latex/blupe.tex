
\documentclass{emulateapj}

\usepackage{amsmath}
\usepackage{graphicx}
\usepackage{url}

\providecommand{\eqn}[1]{eqn.~(\ref{eqn:#1})}
\providecommand{\tab}[1]{Table~\ref{tab:#1}}
\providecommand{\fig}[1]{Figure~\ref{fig:#1}}

\begin{document}

\title{BOSS DR12 Quasar Spectra Calibration Correction}

\author{D. Margala\altaffilmark{1} and D. Kirkby\altaffilmark{1}}
\affil{Frederick Reines Hall, Department of Physics and Astronomy, University of California, Irvine, CA, U.S.A.}
\email{dmargala@uci.edu}

\begin{abstract}
We present a correction for the effects of atmospheric differential refraction of light and guiding offsets on BOSS fiber throughput, which is present due the standard calibration procedure that uses spectrophotometric standard stars that are offset in the focal plane. We calculate focal plane offsets using an ideal guiding scenario for each observation in BOSS DR12 and provide corrections for all quasar targets.
\end{abstract}

\keywords{methods: data analysis --- atmospheric effects --- quasars: general}

\section{Introduction}

Dark energy is cool. We typically charactize dark energy by it's impact on the expansion history of the universe. Measurements of the BAO scale at various redshifts allow us to constrain the expansion history of the univese. BOSS quasar spectra enable a unique measurement of BAO at an important redshift.

Most objects observed in BOSS were positioned in the focal plane to optimize their throughput at 5400\AA. However, quasars were positioned to optimize throughput at 4000\AA~in order to increase throughput in the Lyman-alpha forest. An unavoidable side effect of these offsets is to significantly complicate the spectrophotometric calibration of quasar targets using standard stars that did not have the same offsets applied. In particular, atmospheric differential refraction (ADR) of light and guiding offsets lead to potentially large spectral distortions that are not accounted for in the standard BOSS pipeline.

We describe a model for the miscalibration of quasar spectra and use it to calculate a correction for all BOSS quasar spectra.

\section{Data}

We use data from the BOSS project of SDSS-III.

An object has a unique position (RA, dec) on the sky. A plate \texttt{PLATE} is a set of objects with sky positions transformed to focal plane coordinates for observation at a specific hour angle and optimal throughput wavelength \texttt{LAMBDA\_EFF}. 

An observation corresponds to a unique \texttt{PLATE-MJD}. A target corresponds to a unique \texttt{PLATE-MJD-FIBERID}. An object with multiple observations will correspond to multiple targets.

There are a total of 488824 unique targets (\texttt{PLATE-MJD-FIBERID}) with \texttt{LAMBDA\_EFF == 4000} distributed between 2377 unique observations (\texttt{PLATE-MJD}) of 2340 unique plates (\texttt{PLATE}). Fun fact: there are two plates that have only 1 \texttt{LAMBDA\_EFF == 4000} target.

In addition to the primary BOSS standard stars, a second sample of spectrophotometric standard stars was selected to follow the same offsets in the focal plane as the BOSS quasar targets for a series of plates. These stars were selected and visually inspected to ensure a uniform distribution across the focal plane, in a manner similar to normal spectroscopic standard stars targeted in BOSS. The algorithm for photometric selection applied to this sample was identical to the algorithm for normal spectroscopic standard stars in BOSS, explained in (Dawson 2012). But not exactly because they are chosen with lower priority.

Standard stars offset in the focal plane are identified by bit 20 of \texttt{ANCILLARY\_TARGET2}. We will refer to these objects as ``offset standards''. There are 1782 unique targets (\texttt{PLATE-MJD-FIBERID}) distributed between 161 unique observations (\texttt{PLATE-MJD}) of 159 unique plates (\texttt{PLATE}). There are 79 unique plates (or obs??? need to check) which contain at least 10 offset standards and 20 unique plates with at least 10 offset standards per spectrograph.

The PSF FWHM (\texttt{SEEING50}) distribution for all observations in DR12 is shown in \fig{psf-dist}. There are 79 observations from the early days of the survey that do not have psf information and are excluded from the figure.

\begin{figure}
\centering
\includegraphics[width=75mm]{fig/psf_dist}
\caption{Distribution of psf fwhm for all DR12 observations}
\label{fig:psf-dist}
\end{figure}

\subsubsection{Validation Sample}

For the sample of 79 plates with at least 10 offset standards, we reran the BOSS pipeline using the offset standards for spectrophotometric calibration instead of the normal spectrophotometric standards. This sample provides a vital cross-check for throughput correction model described below. We use, in particular, the sample of 20 plates with at least 10 offset standards in each spectrograph for validation tests. See Appendix X for more details on how the pipeline was modified and where to access the reductions.

\section{Methods}

The basic idea of our correction is to take the ratio of the throughput for a particular fiber as it was positioned in the focal plane and the throughput for that same fiber as if it were positioned with the same offset in the focal plane as the fibers used for calibration.

\subsection{Throughput Calculation}

The throughput for a fiber hole is estimated as the overlap fraction of a 2D Gaussian with variance, $\sigma^2$, offset a distance, $d$, from the center of a circle with diameter $D$ is given by:

\begin{equation}
\label{eq:overlap}
f(\sigma,d,D) = \frac{1}{\sigma ^2}\int_{0}^{D/2}  r e^{-\frac{d^2+r^2}{2 \sigma ^2}} I_{0}\left(\frac{r d}{\sigma ^2}\right) \mathrm{d}r
\end{equation}

where $I_n(z)$ is a modified Bessel function of the first kind. BOSS fibers have a diameter of 2 arcseconds, so we use this value for D throughout. An illustration of an offset 2D Gaussian overlapping with fiberhole is shown in \fig{overlap-demo}. See \fig{overlap-psf}.

\begin{figure}
\centering
\includegraphics[width=84mm]{fig/overlap}
\caption{Overlap fraction example.}
\label{fig:overlap-demo}
\end{figure}

\begin{figure}
\centering
\includegraphics[width=84mm]{fig/overlap-varypsf}
\caption{Varying psf fwhm.}
\label{fig:overlap-psf}
\end{figure}

The flux throughput correction at a wavelength $\lambda$ is defined as:

\begin{equation}
\label{}
C(\lambda,\sigma_{psf},h) = \frac{f(\sigma_{psf},d(\lambda,h,\lambda_{eff}))}{f(\sigma_{psf},d(\lambda,h,\lambda_{cal}))}
\end{equation}

For BOSS quasar targets, $\lambda_{eff} = 4000$~\AA~and $\lambda_{cal} = 5400$~\AA~because the quasars are calibrated using standard stars offset in the focal-plane. 

The offset distance, $d(\lambda,h)$, is determined by the guiding model.

\subsection{Ideal Guiding Model}

An ideal guiding model is used to calculate fiber center offsets. The ideal guiding model determines scale, rotation, and x,y-shift corrections to the plate position in order to minimize offsets of the guide stars from the center of their fiber holes. The inputs to the model are guide star fiber positions at the design hour angle of the plate and the observed hour angle. In practice, guiding is not ideal.

The object centroid light path in the focal plane, $d(\lambda,h)$, for a typical object is shown in \fig{fiber_hole}, where the colored lines correspond to constant $\lambda$ and the dashed gray lines show correspond to constant $h$. The centroid focal plane light path, $d(\lambda=\texttt{LAMBDA\_EFF},h)$ for all targets across an entire plate is shown in \fig{plate_guide}. Guiding offsets have a quadrupole structure across the focal plane.

\begin{figure}
\centering
\includegraphics[width=84mm]{fig/adr-fiber}
\caption{The black circle represents the fiber hole specified by the information in the upper left inset. The colored lines represent paths of light at specific wavelengths (specified in the lower right inset) as a function of observing hour angle for an ideal guiding model. The observing hour angle window shown here is +/- 3 hours from the design hour angle of the plate, indicated in the upper right inset. The dashed gray lines indicate the target's refracted footprint corresponding to the hour angle of actual exposures for this target.}
\label{fig:fiber_hole}
\end{figure}

\begin{figure}
\centering
\includegraphics[width=84mm]{fig/adr-plate}
\caption{The ideal guiding model for an entire plate. The black circle indicates the size of a BOSS plate. The colored lines represent a target's $\lambda_{eff}$ light path spanning the plate's observed hour angle range (indicated in the upper right inset). Red (blue) lines indicate $\lambda_{eff}$ = 5400~\AA~(4000~\AA) targets. The offset distances have been scaled by a factor of 1000.}
\label{fig:plate_guide}
\end{figure}

\subsection{Co-Added Spectra}

The standard pipepline combines individual exposures into a single co-added spectrum. Instead of correcting individual exposures, we use the mean PSF FWHM and observed hour angle of the exposures for an observation and calculate a correction for the coadd. The coadd is a more complicated than this. An alternate simplifying procedure could be to use the PSF FWHM and observed hour angle of the ``best'' exposure as indicated by the pipeline. For the 79 plates without PSF FWHM data, we use the sample median.

\subsection{Correction Parametrization}

We fit a 3 parameter model to the tabulated corrections vector for each target of the form:

\begin{equation}
y = 1 + a_1 \log x/x_0 + a_2 (\log x/x_0)^2
\end{equation}

The $x_0$ parameter gives the crossover point between of the multiplicative correction. In general, the model indicates that BOSS spectra are overestimated blueward and underestimated redward of the crossover point.

\section{Results}

We calculate a thoughput correction for all targets with \texttt{LAMBDA\_EFF == 4000}. The effect of applying the throughput correction for multiple observations of a single object is demonstrated in \fig{adr-vis} and \fig{adr-vis-tpcorr}.

\begin{figure}
\centering
\includegraphics[width=84mm]{fig/adr-vis}
\caption{BOSS quasar with multiple observations. Smoothed with a 21-pixel wide Savitzky-Golay filter of order 2. Compare to Figure 9 of Lee et al (2012).}
\label{fig:adr-vis}
\end{figure}

\begin{figure}
\centering
\includegraphics[width=84mm]{fig/adr-vis-tpcorr}
\caption{BOSS quasar with multiple observations with throughput correction applied. Smoothed with a 21-pixel wide Savitzky-Golay filter of order 2. Compare to Figure 9 of Lee et al (2012).}
\label{fig:adr-vis-tpcorr}
\end{figure}

We use the sample of plates with offset standard stars to validate the correction. See \fig{mag-compare} for a comparison of SDSS imaging magnitudes and synthetic magnitudes before and after correction.

\begin{figure*}
\centering
\includegraphics[width=168mm]{fig/mag-compare}
\caption{The difference between the synthetic photometry computed from BOSS spectra and the measured photometry from the SDSS imaging data for subsets of stars with SDSS $g < 19$ from a sample of 20 plates that have at least 10 \texttt{ANCILLARY\_TARGET2 \& 1<<20 != 0} in each spectrograph. The black, blue, and red lines indicate $X_{synthetic} - X_{PSF}$ offsets for the $g$, $r$, and $i$ filters respectively. The insets in each panel show statistics for the $g$ filter offsets. Top-Left: Spectrophotometric standard stars (Compare directly top-left panel of Figure 8. in Dawson (2012).) Top-Right: \texttt{ANCILLARY\_TARGET2 \& 1<<20 != 0} stars. Bottom-Left: \texttt{ANCILLARY\_TARGET2 \& 1<<20 != 0} stars, from ``blue'' reduction. Bottom-Right: \texttt{ANCILLARY\_TARGET2 \& 1<<20 != 0} stars, with throughput correction applied.}
\label{fig:mag-compare}
\end{figure*}

Finally, we examine the distribution of fitted model parameters to assess the characteristics of the corrections and impact on BOSS spectra.

See \fig{fit-tpcorr-summary} for a summary of model fit results. The plot shows the distribution of mean fitted parameters per plate. Also plot distribution of mean correction per plate?

\begin{figure}
\centering
\includegraphics[width=84mm]{fig/fit-tpcorr-summary}
\caption{Summary of model fits to throughput correction vectors.}
\label{fig:fit-tpcorr-summary}
\end{figure}

\section{Conclusion}

The corrections proposed here seem reasonable.

These results also demonstrate the limitations of attempting to correct the quasar spectrophotometric calibration as a post-processing step after the pipeline, where many simplifying assumptions are required. We believe that in order to achieve better results, it is necessary and desirable to use as much of the existing pipeline calibration algorithms as possible but with spectrophotometric standards observed with the same focal-plane offsets as the quasar targets. In short of including spectrophotometric standards with focal plane offsets, the throughput corrections described here can be applied as a preprocessing step instead of a postprocessing as it has been done here.

\acknowledgments

We are grateful to ....

\appendix

\section{Data Access}

Download the throughput corrections here. Apply corrections to observations by doing this. The code used to calculate the corrections is here.

% \begin{thebibliography}{}

% \end{thebibliography}

\end{document}
