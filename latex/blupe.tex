%%
%% Beginning of file 'sample.tex'
%%
%% Modified 2005 December 5
%%
%% This is a sample manuscript marked up using the
%% AASTeX v5.x LaTeX 2e macros.

%% The first piece of markup in an AASTeX v5.x document
%% is the \documentclass command. LaTeX will ignore
%% any data that comes before this command.

%% The command below calls the preprint style
%% which will produce a one-column, single-spaced document.
%% Examples of commands for other substyles follow. Use
%% whichever is most appropriate for your purposes.
%%
%%\documentclass[12pt,preprint]{aastex}

%% manuscript produces a one-column, double-spaced document:

%% \documentclass[manuscript]{aastex}

%% preprint2 produces a double-column, single-spaced document:

\documentclass[preprint2]{aastex}

%% Sometimes a paper's abstract is too long to fit on the
%% title page in preprint2 mode. When that is the case,
%% use the longabstract style option.

%% \documentclass[preprint2,longabstract]{aastex}

%% If you want to create your own macros, you can do so
%% using \newcommand. Your macros should appear before
%% the \begin{document} command.
%%
%% If you are submitting to a journal that translates manuscripts
%% into SGML, you need to follow certain guidelines when preparing
%% your macros. See the AASTeX v5.x Author Guide
%% for information.

%% You can insert a short comment on the title page using the command below.

\usepackage{amsmath}
\usepackage{graphicx}
\usepackage{url}

%\usepackage[a4paper,margin=2cm,footskip=1cm]{geometry}

\providecommand{\eqn}[1]{eqn.~(\ref{eqn:#1})}
\providecommand{\tab}[1]{Table~\ref{tab:#1}}
\providecommand{\fig}[1]{Figure~\ref{fig:#1}}

%% This is the end of the preamble.  Indicate the beginning of the
%% paper itself with \begin{document}.

\begin{document}

\title{BOSS DR12 Quasar Spectra Calibration Correction}

%% Use \author, \affil, and the \and command to format
%% author and affiliation information.
%% Note that \email has replaced the old \authoremail command
%% from AASTeX v4.0. You can use \email to mark an email address
%% anywhere in the paper, not just in the front matter.
%% As in the title, use \\ to force line breaks.

\author{D. Margala\altaffilmark{1} and D. Kirkby\altaffilmark{1}}
\affil{Frederick Reines Hall, Department of Physics and Astronomy, University of California, Irvine, CA, U.S.A.}

\begin{abstract}
We present a correction for the effects of atmospheric differential refraction of light and guiding offsets, using focal-plane offsets calculated assuming an ideal guiding scenario for individual observations, of quasar all BOSS DR12 quasar targets which is present due the standard calibration procedure with spectrophotometric standard stars that are offset in the focal plane.
\end{abstract}

%% Keywords should appear after the \end{abstract} command. The uncommented
%% example has been keyed in ApJ style. See the instructions to authors
%% for the journal to which you are submitting your paper to determine
%% what keyword punctuation is appropriate.

\keywords{globular clusters: general --- globular clusters: individual(NGC 6397,
NGC 6624, NGC 7078, Terzan 8}

%% From the front matter, we move on to the body of the paper.
%% In the first two sections, notice the use of the natbib \citep
%% and \citet commands to identify citations.  The citations are
%% tied to the reference list via symbolic KEYs. The KEY corresponds
%% to the KEY in the \bibitem in the reference list below. We have
%% chosen the first three characters of the first author's name plus
%% the last two numeral of the year of publication as our KEY for
%% each reference.


%% Authors who wish to have the most important objects in their paper
%% linked in the electronic edition to a data center may do so by tagging
%% their objects with \objectname{} or \object{}.  Each macro takes the
%% object name as its required argument. The optional, square-bracket 
%% argument should be used in cases where the data center identification
%% differs from what is to be printed in the paper.  The text appearing 
%% in curly braces is what will appear in print in the published paper. 
%% If the object name is recognized by the data centers, it will be linked
%% in the electronic edition to the object data available at the data centers  
%%
%% Note that for sources with brackets in their names, e.g. [WEG2004] 14h-090,
%% the brackets must be escaped with backslashes when used in the first
%% square-bracket argument, for instance, \object[\[WEG2004\] 14h-090]{90}).
%%  Otherwise, LaTeX will issue an error. 

\section{Introduction}

Most BOSS targets are drilled to optimize their throughput at 5400\AA. However, quasar targets are optimized for throughput at 4000\AA~by introducing focal plane offsets. An unavoidable side effect of these offsets is to significantly complicate the spectrophotometric calibration of quasar targets using standard stars that do not have the same offsets applied. In particular, atmospheric differential refraction (ADR) and guiding offsets lead to potentially large spectral distortions that are not accounted for in the standard BOSS pipeline, leading to ``calibrated'' exposure sequences such as the one illustrated in Figure (insert figure here).

The effects of ADR are relatively well understood and straightforward to model.\footnote{\url{http://emtoolbox.nist.gov/Wavelength/Ciddor.asp}}\footnote{\url{http://mintaka.sdsu.edu/GF/explain/atmos_refr/models/Cassini.html}}

\section{Data}

\subsection{BOSS DR12}

\subsubsection{Ancillary Program}

We propose to target spectrophotometric standard stars using the same focal-plane offset that is applied to quasar targets. The objectives of this proposal are to:

\begin{itemize}
\item Calibrate quasar spectra in future observations using the existing pipeline spectrophotometric calibration algorithms combined with offset standard stars. The resulting quasar sample potentially improves any analysis of quasar spectra, and offers a straightforward method for studying the systematics of the present calibration scheme.
\item Use existing models of the relatively straightforward ADR effects to isolate the effects of guiding errors by comparing standard stars observed with and without offsets in future observations. This offers a potentially valuable new approach for directly monitoring the effects of guiding errors on throughput.
\item Develop a combined global spectrophotometric calibration that uses standard stars with and without offsets to calibrate all targets on a plate. This would require developing and testing new algorithms but has the potential to improve the calibration of all targets, not just quasars.
\end{itemize}

We are requesting 20 fibers per plate (~3 per square degree) to match the current number of spectrophotometric standard stars without focal plane offsets that are targeted per plate. Note that the requested density is determined by the expected scale of spatial variations in the spectrophotometric calibration rather than the density of the quasar targets.

The number of spectrophotometric standard stars required is determined by the large-scale throughput variation across a plate. Ideally, we would like to use the existing targeting algorithm to target an additional ~3 spectrophotometric standard stars per square degree to achieve the same plate coverage (20 per plate) as the existing spectrophotometric standard star targeting. We do not consider incompleteness to be a concern. The program would benefit by having a spatial homogeneity similar to the existing spectrophotometric standard star targeting distribution over a plate. The target density on the sky for the full list of targets used by the existing spectrophotometric standard star selection is shown in Figure 8.

The full list of targets used by the existing spectrophotometric standards star selection is available at %\url{https://trac.sdss3.org/wiki/BOSS/target/releases/main008}

At the end of the day we ended up with approximately 79 plates with targets.

\subsubsection{Alternate Reduction}

Modify the pipeline input files so that the ancillary targets are labeled as spectrophotometric standards. Run the pipeline on the ancillary plates. Results are available at blahblahblah.

\subsubsection{Other Samples}

Any comparison of targets observed in both SDSS/BOSS?

Revisit viability of using failed quasar targets as a possible sample?

Since roughly half of the quasar targets in the BOSS DR9 sample are actually stars, it is possible that sufficient observations of spectrophotometric standard stars in fiber holes with quasar focal plane offsets already exist. To investigate this, we examine the color distributions of objects targeted as quasars but classified as stars by the spectroscopic pipeline and compare the results to the spectrophotometric standard star targets. Figure 7 illustrates that there is a small fraction of failed quasars that have bright enough r-band magnitudes but fewer than 0.2\% pass the color space selection cuts for BOSS spectrophotometric standard stars. We estimate that we would need ~35\% passing the color space selection cuts (assuming ideal sky coverage) in order to match the number of targeted spectrophotometric standards without focal plane offsets, so clearly the existing failed quasar sample is not sufficient to meet the goals of this proposal.


\section{Methods}

The correction for each individual target is calculated by estimating the throughput using the actual fiber hole positions and assuming they were positioned using the same focal-plane offset as the calibration targets.

\subsection{Throughput Calculation}

The throughput for a target is estimated as the overlap fraction, of a 2D Gaussian with variance, $\sigma^2$, offset a distance, $d$, from the center of a circle with diameter $D$ is given by:

\begin{equation}
\label{eq:overlap}
f(\sigma,d,D) = \frac{1}{\sigma ^2}\int_{0}^{D/2}  r e^{-\frac{d^2+r^2}{2 \sigma ^2}} I_{0}\left[\frac{r d}{\sigma ^2}\right] \mathrm{d}r
\end{equation}

where $I_n(z)$ is a modified Bessel function of the first kind. BOSS fibers have a diameter of 2 arcseconds, so we use this value for D throughout. See \fig{overlap}.

\begin{figure}
\centering
\includegraphics[width=84mm]{fig/overlap}
\caption{Overlap fraction example.}
\label{fig:overlap}
\end{figure}

The flux throughput correction at a wavelength $\lambda$ is defined as:

\begin{equation}
\label{}
C(\lambda,\sigma_{psf},h) = \frac{f(\sigma_{psf},d(\lambda,h,\lambda_{eff}))}{f(\sigma_{psf},d(\lambda,h,\lambda_{cal}))}
\end{equation}

For BOSS quasar targets, $\lambda_{eff} = 4000$~\AA~and $\lambda_{cal} = 5400$~\AA~because the quasars are calibrated using standard stars offset in the focal-plane. 

The offset distance, $d$, is determined by the guiding model.

\subsection{Ideal Guiding Model}

A guiding model is used to calculate fiber center offsets instead of the formulas above. The guide model determines scale, rotation, and shift corrections to the plate position in order to minimize offsets of the guide stars from the center of their drilled fiber holes. The light path from a target for several different wavelengths for a typical targer is shown in \fig{fiber_hole}. The light path pattern across an entire plate is shown in \fig{plate_guide}. Guiding offsets have a quadrupole structure across the focal plane.

\begin{figure}
\centering
\includegraphics[width=84mm]{fig/adr-fiber}
\caption{The black circle represents the fiber hole specified by the information in the upper left inset. The colored lines represent paths of light at specific wavelengths (specified in the lower right inset) as a function of observing hour angle for an ideal guiding model. The observing hour angle window shown here is +/- 3 hours from the design hour angle of the plate, indicated in the upper right inset. The dashed gray lines indicate the target's refracted footprint corresponding to the hour angle of actual exposures for this target.}
\label{fig:fiber_hole}
\end{figure}

\begin{figure}
\centering
\includegraphics[width=84mm]{fig/adr-plate}
\caption{The ideal guiding model for an entire plate. The black circle indicates the size of a BOSS plate. The colored lines represent a target's $\lambda_{eff}$ light path spanning the plate's observed hour angle range (indicated in the upper right inset). Red (blue) lines indicate $\lambda_{eff}$ = 5400~\AA~(4000~\AA) targets. The offset distances have been scaled by a factor of 1000.}
\label{fig:plate_guide}
\end{figure}

\subsection{Correction Parametrization}

We fit a model to the tabulated corrections.

\begin{equation}
y = 1 + a_1 \log x/x_0 + a_2 (\log x/x_0)^2
\end{equation}

See \fig{param} for example fit results.

\begin{figure*}
\centering
\includegraphics[width=168mm]{fig/param}
\caption{Example parametrization fit results.}
\label{fig:param}
\end{figure*}

\subsection{Validation}

Clearly demonstarte calibration issue. 

Individual plate comparison.

\section{Results}

The Ideal Guiding model gives a correction at each 4000~\AA~fiber. 

\fig{correction} compares the model predications to the actual ADR effect determined from the ``blue standard'' ancillary science program.

\begin{figure*}
\centering
\includegraphics[width=168mm]{fig/tpratio}
\caption{The corrections for all $\lambda_{eff}$ = 4000~\AA~targets on a plate using the mean observing hour angle and mean PSF size. The transparent lines correspond to correction curves for individual targets while the thick opaque lines correspond the coadded spectrum ratios of the standard and alternate reductions for ancillary targets on this plate. The blue and red colors indicate fibers 1-500 and 501-1000 respectively.}
\label{fig:correction}
\end{figure*}

\section{Conclusion}

The corrections proposed here seem reasonable.

These results also demonstrate the limitations of attempting to correct the quasar spectrophotometric calibration as a post-processing step after of the pipeline, where many simplifying assumptions are required. We believe that in order to achieve better results, it is necessary and desirable to use as much of the existing pipeline calibration algorithms as possible but with spectrophotometric standards observed with the same focal-plane offsets as the quasar targets. In short of including spectrophotometric standards with focal plane offsets, the throughput corrections described here can be applied as a preprocessing step instead of a postprocessing as it has been done here.


%% If you wish to include an acknowledgments section in your paper,
%% separate it off from the body of the text using the \acknowledgments
%% command.

%% Included in this acknowledgments section are examples of the
%% AASTeX hypertext markup commands. Use \url without the optional [HREF]
%% argument when you want to print the url directly in the text. Otherwise,
%% use either \url or \anchor, with the HREF as the first argument and the
%% text to be printed in the second.

\acknowledgments

We are grateful to V. Barger, T. Han, and R. J. N. Phillips for
doing the math in section~\ref{bozomath}.
More information on the AASTeX macros package is available \\ at
\url{http://www.aas.org/publications/aastex}.
For technical support, please write to
\email{aastex-help@aas.org}.

%% To help institutions obtain information on the effectiveness of their
%% telescopes, the AAS Journals has created a group of keywords for telescope
%% facilities. A common set of keywords will make these types of searches
%% significantly easier and more accurate. In addition, they will also be
%% useful in linking papers together which utilize the same telescopes
%% within the framework of the National Virtual Observatory.
%% See the AASTeX Web site at http://aastex.aas.org/
%% for information on obtaining the facility keywords.

%% After the acknowledgments section, use the following syntax and the
%% \facility{} macro to list the keywords of facilities used in the research
%% for the paper.  Each keyword will be checked against the master list during
%% copy editing.  Individual instruments or configurations can be provided 
%% in parentheses, after the keyword, but they will not be verified.

{\it Facilities:} \facility{Nickel}, \facility{HST (STIS)}, \facility{CXO (ASIS)}.

%% Appendix material should be preceded with a single \appendix command.
%% There should be a \section command for each appendix. Mark appendix
%% subsections with the same markup you use in the main body of the paper.

%% Each Appendix (indicated with \section) will be lettered A, B, C, etc.
%% The equation counter will reset when it encounters the \appendix
%% command and will number appendix equations (A1), (A2), etc.

\appendix

\section{Data Access}

%% The reference list follows the main body and any appendices.
%% Use LaTeX's thebibliography environment to mark up your reference list.
%% Note \begin{thebibliography} is followed by an empty set of
%% curly braces.  If you forget this, LaTeX will generate the error
%% "Perhaps a missing \item?".
%%
%% thebibliography produces citations in the text using \bibitem-\cite
%% cross-referencing. Each reference is preceded by a
%% \bibitem command that defines in curly braces the KEY that corresponds
%% to the KEY in the \cite commands (see the first section above).
%% Make sure that you provide a unique KEY for every \bibitem or else the
%% paper will not LaTeX. The square brackets should contain
%% the citation text that LaTeX will insert in
%% place of the \cite commands.

%% We have used macros to produce journal name abbreviations.
%% AASTeX provides a number of these for the more frequently-cited journals.
%% See the Author Guide for a list of them.

%% Note that the style of the \bibitem labels (in []) is slightly
%% different from previous examples.  The natbib system solves a host
%% of citation expression problems, but it is necessary to clearly
%% delimit the year from the author name used in the citation.
%% See the natbib documentation for more details and options.

\begin{thebibliography}{}
\bibitem[Auri\`ere(1982)]{aur82} Auri\`ere, M.  1982, \aap,
    109, 301
\bibitem[Canizares et al.(1978)]{can78} Canizares, C. R.,
    Grindlay, J. E., Hiltner, W. A., Liller, W., \&
    McClintock, J. E.  1978, \apj, 224, 39
\bibitem[Djorgovski \& King(1984)]{djo84} Djorgovski, S.,
    \& King, I. R.  1984, \apjl, 277, L49
\bibitem[Hagiwara \& Zeppenfeld(1986)]{hag86} Hagiwara, K., \&
    Zeppenfeld, D.  1986, Nucl.Phys., 274, 1
\bibitem[Harris \& van den Bergh(1984)]{har84} Harris, W. E.,
    \& van den Bergh, S.  1984, \aj, 89, 1816
\bibitem[H\`enon(1961)]{hen61} H\'enon, M.  1961, Ann.d'Ap., 24, 369
\bibitem[Heiles \& Troland(2003)]{heiles03} Heiles, C. \& Troland, T. H., 2003, \apjs, preprint doi:10.1086/381753
\bibitem[Kim, Ostricker, \& Stone(2003)]{kim03} Kim, W.-T.,  Ostriker, E., \& Stone, J. M., 2003, \apj, 599, 1157
\bibitem[King(1966)]{kin66}  King, I. R.  1966, \aj, 71, 276
\bibitem[King(1975)]{kin75}  King, I. R.  1975, Dynamics of
    Stellar Systems, A. Hayli, Dordrecht: Reidel, 1975, 99
\bibitem[King et al.(1968)]{kin68}  King, I. R., Hedemann, E.,
    Hodge, S. M., \& White, R. E.  1968, \aj, 73, 456
\bibitem[Kron et al.(1984)]{kro84} Kron, G. E., Hewitt, A. V.,
    \& Wasserman, L. H.  1984, \pasp, 96, 198
\bibitem[Lynden-Bell \& Wood(1968)]{lyn68} Lynden-Bell, D.,
    \& Wood, R.  1968, \mnras, 138, 495
\bibitem[Newell \& O'Neil(1978)]{new78} Newell, E. B.,
    \& O'Neil, E. J.  1978, \apjs, 37, 27
\bibitem[Ortolani et al.(1985)]{ort85} Ortolani, S., Rosino, L.,
    \& Sandage, A.  1985, \aj, 90, 473
\bibitem[Peterson(1976)]{pet76} Peterson, C. J.  1976, \aj, 81, 617
\bibitem[Rudnick et al.(2003)]{rudnick03} Rudnick, G. et al., 2003, \apj, 599, 847
\bibitem[Spitzer(1985)]{spi85} Spitzer, L.  1985, Dynamics of
    Star Clusters, J. Goodman \& P. Hut, Dordrecht: Reidel, 109
\bibitem[Treu et al.(2003)]{treu03} Treu, T. et al., 2003, \apj, 591, 53
\end{thebibliography}

\clearpage

%% Use the figure environment and \plotone or \plottwo to include
%% figures and captions in your electronic submission.
%% To embed the sample graphics in
%% the file, uncomment the \plotone, \plottwo, and
%% \includegraphics commands
%%
%% If you need a layout that cannot be achieved with \plotone or
%% \plottwo, you can invoke the graphicx package directly with the
%% \includegraphics command or use \plotfiddle. For more information,
%% please see the tutorial on "Using Electronic Art with AASTeX" in the
%% documentation section at the AASTeX Web site, http://aastex.aas.org/
%%
%% The examples below also include sample markup for submission of
%% supplemental electronic materials. As always, be sure to check
%% the instructions to authors for the journal you are submitting to
%% for specific submissions guidelines as they vary from
%% journal to journal.

\end{document}
